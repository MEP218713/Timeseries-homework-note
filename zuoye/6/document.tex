\documentclass[11pt,a4paper]{ctexart}
\usepackage{fontspec}
\defaultfontfeatures{Mapping=tex-text}
\usepackage{xunicode}
\usepackage{xltxtra}
%\setmainfont{???}
\usepackage{amsmath}
\usepackage{amsfonts}
\usepackage{amssymb}
\usepackage{graphicx}
\usepackage{amsthm}
\usepackage{array}
\usepackage{float}   %{H}
\usepackage{booktabs}  %\toprule[1.5pt]
\usepackage[titletoc]{appendix}
%===================%插入代码需要的控制
\usepackage{listings}
\usepackage{xcolor}
\lstset{
	numbers=left, 
	numberstyle= \tiny, 
	keywordstyle= \color{ blue!70},
	commentstyle= \color{red!50!green!50!blue!50}, 
	frame=shadowbox, % 阴影效果
	rulesepcolor= \color{ red!20!green!20!blue!20} ,
	escapeinside=``, % 英文分号中可写入中文
} 
%===================%
\usepackage[left=2cm,right=2cm,top=2cm,bottom=2cm]{geometry}

\newtheorem{theorem}{定理}
\newtheorem{definition}{定义}
\newtheorem*{solution}{解}
\newtheorem{practice}{题}

\title{Time Series HomeWork (6)}
\author{钟瑜 \quad 222018314210044}
\date{\today}
\begin{document}
\maketitle
\pagestyle{plain}%设置页码
\begin{enumerate}
%================================================================%	
	
\item[1.]$ X_t=\frac{1}{6}X_{t-1}+\frac{1}{6}X_{t-2}+\epsilon_t $, $ \left\lbrace \epsilon_t\right\rbrace \sim WN(0,0.25) $

求它的平稳解以及自协方差函数$ \gamma_k $
\begin{solution}
平稳解的一般形式为:
\begin{equation}
X_t=\sum_{j=0}^{\infty}\psi_j\epsilon_{t-j}
\end{equation}
先只需确定$ \psi_j $.由题目可知$ p=2 $,$ a_1=a_2=\frac{1}{6}$.而
\[
\psi_k =
\begin{cases}
	0 &  k<0 \\
	1 &  k=0 \\
	\sum_{j=1}^p a_j \psi_{k-j} & k\geq 1
\end{cases}
\]

当$ k\geq 1 $时,$ \psi_k=\frac{1}{6}(\psi_{k-1}+\psi_{k-2}) $,解该差分方程得
\begin{equation}
\psi_k=\frac{3}{5}(\frac{1}{2})^k+\frac{2}{5}(\frac{-1}{3})^k 
\end{equation}
故平稳解为:
\begin{equation}
	X_t=\sum_{j=0}^{\infty}\left[ \frac{3}{5}(\frac{1}{2})^j+\frac{2}{5}(\frac{-1}{3})^j \right] \epsilon_{t-j}
\end{equation}
自协方差函数$ \gamma_k $为:
\begin{equation}
\begin{aligned}
\gamma_k &= \sigma^2\sum_{j=1}^{\infty} \psi_j\psi_{j+k}\\
&= 0.25\sum_{j=1}^{\infty} \left[ \frac{3}{5}(\frac{1}{2})^j+\frac{2}{5}(\frac{-1}{3})^j \right]\left[ \frac{3}{5}(\frac{1}{2})^{j+k}+\frac{2}{5}(\frac{-1}{3})^{j+k} \right]\\
&= \frac{1}{100}\left[ 3(\frac{1}{2})^{k-2}+\frac{4}{7}(\frac{-1}{3})^{k-2}+\frac{9}{7}(\frac{1}{2})^{k-2}+2(\frac{-1}{3})^{k-2}\right] 
\end{aligned}
\end{equation}
\end{solution}
\item[2.]设$ \left\lbrace \epsilon_t \right\rbrace  $是$ WN(\mu,\sigma^2) $, $ a_1,a_2,...,a_p $满足最小相位条件.求非中心化AR(p)模型:
\begin{equation}
X_t=a_0+\sum_{j=1}^{p}a_jX_{t-j}+\epsilon_t\;,t\in\mathbb{Z}
\end{equation}
的平稳解和通解.
\begin{solution}
\begin{equation}
\begin{aligned}
X_t &= a_0+\sum_{j=1}^{p}a_jX_{t-j}+\epsilon_t\\
&= \sum_{j=1}^{p}a_jX_{t-j}+a_0+\epsilon_t
\end{aligned}
\end{equation}
而$ 	A(\mathcal{B})X_t=\epsilon_t+a_0 $,故
\begin{equation}
\begin{aligned}
X_t &= A^{-1}(\mathcal{B})A(\mathcal{B})X_t\\
&= A^{-1}(\mathcal{B})(\epsilon_t+a_0)\\
&= \sum_{j=0}^{\infty}\psi_j\epsilon_{t-j}+a_0\\
\end{aligned}
\end{equation}
上式即为平稳解.

通解为:
\begin{equation}
	X_t=a_0+\sum_{j=0}^{\infty}\psi_j\epsilon_{t-j}+\sum_{j=1}^{k}\sum_{l=0}^{r(j)-1}U_{l,j}t^lz_j^{-t},t\in\mathbb{Z}
\end{equation}
\end{solution}

%=================================================================%
\end{enumerate}
\end{document}