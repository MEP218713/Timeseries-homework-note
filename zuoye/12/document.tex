\documentclass[11pt,a4paper]{ctexart}
\usepackage{fontspec}
\defaultfontfeatures{Mapping=tex-text}
\usepackage{xunicode}
\usepackage{xltxtra}
%\setmainfont{???}
\usepackage{amsmath}
\usepackage{amsfonts}
\usepackage{amssymb}
\usepackage{graphicx}
\usepackage{amsthm}
\usepackage{array}
\usepackage{float}   %{H}
\usepackage{booktabs}  %\toprule[1.5pt]
\usepackage[titletoc]{appendix}
%===================%插入代码需要的控制
\usepackage{listings}
\usepackage{xcolor}
\setmonofont{Consolas}%字体
\lstset{
	keywordstyle= \color{ blue!70},
	commentstyle= \color{red!50!green!50!blue!50}, 
	frame=shadowbox, % 阴影效果
	rulesepcolor= \color{ red!20!green!20!blue!20} ,
	escapeinside=``,% 英文分号中可写入中文
	breaklines=true,
	basicstyle=\ttfamily 
} 
%===================%
\usepackage[left=2cm,right=2cm,top=2cm,bottom=2cm]{geometry}

\newtheorem{theorem}{定理}
\newtheorem{definition}{定义}
\newtheorem*{solution}{解}
\newtheorem{practice}{题}

\title{Time Series HomeWork (12)}
\author{钟瑜 \quad 222018314210044}
\date{\today}
\begin{document}
\maketitle
\pagestyle{plain}%设置页码
\begin{enumerate}
%================================================================%	
	
\item[1.] 假设时间序列服从AR(2)过程
$$X_t=a_1X_{t-1}+a_2X_{t-2}+\epsilon_t,\left\lbrace \epsilon_t\right\rbrace \sim WN(0,\sigma^2)$$
其样本自协方差系数$$\hat{\gamma}_0=1382.2,\hat{\gamma}_1=1114.4,\hat{\gamma}_2=591.73,\hat{\gamma}_3=96.216$$
试用这些估计值给出$ a_1,a_2,\sigma^2 $的yule-walker估计,给出$ a_1,a_2 $的95\%置信区间.
\begin{solution}
R代码如下:
\end{solution}

\begin{lstlisting}[language=r]
> g12<-matrix(c(1114.4,591.73),byrow = TRUE,ncol=1)
> g0110<-matrix(c(1382.2,591.73,591.73,1382.2),byrow = TRUE,ncol=2)
> g0110n<-solve(g0110)
> a12<-g0110n%*%g12
> a12
          [,1]
[1,] 0.7627729
[2,] 0.1015587
> sigma2<-1382.2-t(a12)%*%g12
> sigma2
         [,1]
[1,] 472.0706

#==============求sigmajj=================#
> sigma2<-as.numeric(sigma2)   
> sigmajj<-sigma2*g0110n
> sigmajj
           [,1]       [,2]
[1,]  0.4181775 -0.1790249
[2,] -0.1790249  0.4181775
\end{lstlisting}
故$$ \hat{a}_1= 0.7627729,\hat{a}_2=0.1015587,\sigma^2=472.0706$$

设$ x_1, x_2, ..., x_N  $为观测数据,于是$ a_j $ 的近似
95\% 置信区间为$$ [\hat{a}_j-1.96\sqrt{\sigma_{jj/N}},\hat{a}_j+1.96\sqrt{\sigma_{jj/N}}]$$
i.e.
$$[0.7627729-1.96\sqrt{0.4181775/N},0.7627729-1.96\sqrt{0.4181775/N}]$$
$$[0.1015587-1.96\sqrt{0.4181775/N},0.1015587-1.96\sqrt{0.4181775/N}]$$
%=================================================================%
\end{enumerate}
\end{document}