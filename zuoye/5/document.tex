\documentclass[11pt,a4paper]{ctexart}
\usepackage{fontspec}
\defaultfontfeatures{Mapping=tex-text}
\usepackage{xunicode}
\usepackage{xltxtra}
%\setmainfont{???}
\usepackage{amsmath}
\usepackage{amsfonts}
\usepackage{amssymb}
\usepackage{graphicx}
\usepackage{amsthm}
\usepackage{array}
\usepackage{float}   %{H}
\usepackage{booktabs}  %\toprule[1.5pt]
\usepackage[titletoc]{appendix}
%===================%插入代码需要的控制
\usepackage{listings}
\usepackage{xcolor}
\lstset{
	numbers=left, 
	numberstyle= \tiny, 
	keywordstyle= \color{ blue!70},
	commentstyle= \color{red!50!green!50!blue!50}, 
	frame=shadowbox, % 阴影效果
	rulesepcolor= \color{ red!20!green!20!blue!20} ,
	escapeinside=``, % 英文分号中可写入中文
} 
%===================%
\usepackage[left=2cm,right=2cm,top=2cm,bottom=2cm]{geometry}

\newtheorem{theorem}{定理}
\newtheorem{definition}{定义}
\newtheorem*{solution}{解}
\newtheorem{practice}{题}

\title{Time Series HomeWork (5)}
\author{钟瑜 \quad 222018314210044}
\date{\today}
\begin{document}
\maketitle
\pagestyle{plain}%设置页码
\begin{enumerate}
%================================================================%	
	
\item[1.]求齐次差分方程$ X_t=\sqrt{2}X_{t-1}-X_{t-2} $的通解.
\begin{solution}
特征多项式为
\begin{equation}
A(x)=1-\sqrt{2}x+x^2
\end{equation}
有两个复根:$ \frac{\sqrt{2}}{2}(1+i) $和$ \frac{\sqrt{2}}{2}(1-i) $.则$ [\frac{\sqrt{2}}{2}(1+i)]^{-t} $和$[\frac{\sqrt{2}}{2}(1-i)]^{-t} $为方程的两个解.

通解为
\begin{equation}
X_t=U_1[\frac{\sqrt{2}}{2}(1+i)]^{-t}+U_2[\frac{\sqrt{2}}{2}(1-i)]^{-t},\;t\in\mathbb{Z}
\end{equation}
其中r.v. $ U_1,U_2 $为由$ \left\lbrace X_t\right\rbrace  $的初值$X_0,X_1 $唯一决定.
\end{solution}

\item[2.]设$ \epsilon_t \sim WN(0,\sigma^2)$,现有一阶非齐次差分方程方程$ X_t-0.5X_{t-1}=\epsilon_t $.
\begin{enumerate}
	\item[1)]判断并证明$ X_t^{(0)}=\sum_{j=0}^{\infty }(0.5)^j \epsilon_{t-j} $是否为上述非齐次差分方程的一个特定解?
	\item[2)]如上定义的$  X_t^{(0)} $是(弱)平稳序列吗?为啥?
	\item[3)]求出上述一阶非齐次差分方程的通解$ X_t $?并且写出当$ t\rightarrow\infty $时,该通解$ X_t $的收敛性质?
\end{enumerate}
\begin{solution}
\begin{enumerate}
	\item[1)]是.证明如下:
	
	方程的特征多项式为
\begin{equation}
		A(x)=1-0.5x
\end{equation}
则
\begin{equation}
\begin{aligned}
	A(\mathcal{B})(X_t- X_t^{(0)})&=(1-0.5\mathcal{B})(X_t- X_t^{(0)})\\
	&=(1-0.5\mathcal{B})(0.5X_{t-1}+\epsilon_t-\sum_{j=0}^{\infty }(0.5)^j \epsilon_{t-j} )\\
	&=0.5X_{t-1}+\epsilon_t-\sum_{j=0}^{\infty }(0.5)^j \epsilon_{t-j}-0.25X_{t-2}-0.5\epsilon_{t-1}+0.5\sum_{j=0}^{\infty }(0.5)^j \epsilon_{t-j-1}\\
	&=0.5X_{t-1}-0.25X_{t-2}-\sum_{j=1}^{\infty }(0.5)^j \epsilon_{t-j}+0.5\sum_{j=1}^{\infty }(0.5)^j \epsilon_{t-j-1}\\
	&=0.5X_{t-1}-0.25X_{t-2}-0.5\epsilon_{t-1}\\
	&=0.5\epsilon_{t-1}-0.5\epsilon_{t-1}\\
	&=0\\
\end{aligned}
\end{equation}
其中$ t\in\mathbb{Z} $

	\item[2)]如上定义的$  X_t^{(0)} $是(弱)平稳序列.因为为零均值同方差的白噪声线性组合,其实系数列$ \sum_{j=0}^{\infty }(0.5)^j $平方可和,故$  X_t^{(0)} $为线性平稳序列.
	
	\item[3)]方程的特征多项式为
	\begin{equation}
		A(x)=1-0.5x
	\end{equation}
有单根$ x=2 $.那么$ 2^{-t} $为方程的解.

通解为
\begin{equation}
	X_t=X_t^{(0)}+U2^{-1}=\sum_{j=0}^{\infty }(0.5)^j \epsilon_{t-j}+U2^{-1},\;t\in\mathbb{Z}
\end{equation}
其中r.v. $ U$为由$ \left\lbrace X_t\right\rbrace  $的初值$X_0 $唯一决定.

当$ t\rightarrow\infty $时,该通解$ X_t $的收敛到特解$\sum_{j=0}^{\infty }(0.5)^j \epsilon_{\infty-j}=\epsilon_{\infty}\sum_{j=0}^{\infty }(0.5)^j = \epsilon_{\infty}*0=0$
\end{enumerate}
\end{solution}
%=================================================================%
\end{enumerate}
\end{document}