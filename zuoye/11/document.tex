\documentclass[11pt,a4paper]{ctexart}
\usepackage{fontspec}
\defaultfontfeatures{Mapping=tex-text}
\usepackage{xunicode}
\usepackage{xltxtra}
%\setmainfont{???}
\usepackage{amsmath}
\usepackage{amsfonts}
\usepackage{amssymb}
\usepackage{graphicx}
\usepackage{amsthm}
\usepackage{array}
\usepackage{float}   %{H}
\usepackage{booktabs}  %\toprule[1.5pt]
\usepackage[titletoc]{appendix}
%===================%插入代码需要的控制
\usepackage{listings}
\usepackage{xcolor}
\setmonofont{Consolas}%字体
\lstset{
	keywordstyle= \color{ blue!70},
	commentstyle= \color{red!50!green!50!blue!50}, 
	frame=shadowbox, % 阴影效果
	rulesepcolor= \color{ red!20!green!20!blue!20} ,
	escapeinside=``,% 英文分号中可写入中文
	breaklines=true,
	basicstyle=\ttfamily 
} 
%===================%
\usepackage[left=2cm,right=2cm,top=2cm,bottom=2cm]{geometry}

\newtheorem{theorem}{定理}
\newtheorem{definition}{定义}
\newtheorem*{solution}{解}
\newtheorem{practice}{题}

\title{Time Series HomeWork (11)}
\author{钟瑜 \quad 222018314210044}
\date{\today}
\begin{document}
\maketitle
\pagestyle{plain}%设置页码
\begin{enumerate}
%================================================================%	
	
\item[1.] 给定平稳序列$ \left\lbrace X_t \right\rbrace  $的自协方差函数
$ (\gamma_0,\gamma_1,...,\gamma_4)=(5.61,-1.1,0.23,0.43,-0.1) $.
试为$ \left\lbrace X_t \right\rbrace $建立ARMA(2,2)模型.
\begin{solution}
R代码如下:
\end{solution}

1.求AR部分的系数:
\begin{lstlisting}[language=r]
> ga34<-matrix(c(0.43,-0.1),byrow = TRUE,nrow = 2)
> ga34
      [,1]
[1,]  0.43
[2,] -0.10
> ga2132<-matrix(c(0.23,-1.1,0.43,0.23),byrow = TRUE,ncol = 2)
> ga2132
     [,1]  [,2]
[1,] 0.23 -1.10
[2,] 0.43  0.23
> ga2132ni<-solve(ga2132)
> a12<-ga2132%*%ga34
> a12
       [,1]
[1,] 0.2089
[2,] 0.1619

\end{lstlisting}
则$ a_1=0.2089 $,$ a_2=0.1619 $,
转换为MA模型: $$Y_t=X_t-0.2089X_{t-1}-0.1619X_{t-2}=\epsilon_t+b_1\epsilon_{t-1}+b_2\epsilon_{t-2}$$


2.求MA序列$ \left\lbrace Y_t \right\rbrace  $的自协方差函数$ \left\lbrace \gamma_y(k) \right\rbrace  $:
\begin{lstlisting}[language=r]
> gak0<-matrix(c(5.61,-1.1,0.23,
+                -1.1,5.61,-1.1,
+                0.23,-1.1,5.61),byrow=TRUE,ncol=3)
> gak0
      [,1]  [,2]  [,3]
[1,]  5.61 -1.10  0.23
[2,] -1.10  5.61 -1.10
[3,]  0.23 -1.10  5.61
> gak1<-matrix(c(-1.1,0.23,0.43,
+                5.61,-1.1,0.23,
+                -1.1,5.61,-1.1),byrow=TRUE,ncol=3)
> gak1
      [,1]  [,2]  [,3]
[1,] -1.10  0.23  0.43
[2,]  5.61 -1.10  0.23
[3,] -1.10  5.61 -1.10
> gak2<-matrix(c(0.23,0.43,-0.1,
+                -1.1,0.23,0.43,
+                5.61,-1.1,0.23),byrow=TRUE,ncol=3)
> gak2
      [,1]  [,2]  [,3]
[1,]  0.23  0.43 -0.10
[2,] -1.10  0.23  0.43
[3,]  5.61 -1.10  0.23

> gay0<-a012%*%gak0%*%a012t
> gay1<-a012%*%gak1%*%a012t
> gay2<-a012%*%gak2%*%a012t
> gay0;gay1;gay2
         [,1]
[1,] 6.312563
          [,1]
[1,] -2.090825
           [,1]
[1,] -0.5287003
\end{lstlisting}

3.求MA序列$ \left\lbrace Y_t \right\rbrace  $的系数和$ \sigma^2 $:
\begin{lstlisting}[language=r]
## Given \gamma_0, \gamma_1, \dots, \gamma_q,
## Solve MA(q) coefficients b_1, \dots, b_q, \sigma^2
## Using Li Lei's algorithm.
## Input: gms -- \gamma_0, \gamma_1, \dots, \gamma_q
> ma.solve <- function(gms, k=100){
	+   q <- length(gms)-1
	+   if(q==1){
		+     rho1 <- gms[2] / gms[1]
		+     b <- (1 - sqrt(1 - 4*rho1^2))/(2*rho1)
		+     s2 <- gms[1] / (1 + b^2)
		+     return(list(b=b, s2=s2))
		+   }
	+   A <- matrix(0, nrow=q, ncol=q)
	+   for(j in seq(2,q)){
		+     A[j-1,j] <- 1
		+   }
	+   cc <- numeric(q); cc[1] <- 1
	+   gamma0 <- gms[1]
	+   gammas <- numeric(q+k)
	+   gammas[1:(q+1)] <- gms
	+   gamq <- gms[-1]
	+   Gammak <- matrix(0, nrow=k, ncol=k)
	+   for(ii in seq(k)){
		+     for(jj in seq(k)){
			+       Gammak[ii,jj] <- gammas[abs(ii-jj)+1]
			+     }
		+   }
	+   Omk <- matrix(0, nrow=q, ncol=k)
	+   for(ii in seq(q)){
		+     for(jj in seq(k)){
			+       Omk[ii,jj] <- gammas[ii+jj-1+1]
			+     }
		+   }
	+   PI <- Omk %*% solve(Gammak, t(Omk))
	+   s2 <- gamma0 - c(t(cc) %*% PI %*% cc)
	+   b <- 1/s2 * c(gamq - A %*% PI %*% cc)
	+   return(list(b=b, s2=s2))
	+ }

> gms<-c(gay0,gay1,gay2)
> ma.solve(gms,k=100)
$b
[1] -0.4454071 -0.1012276

$s2
[1] 5.222888
\end{lstlisting}

故我们得到ARMA(2,2)模型
$$ X_t=0.2089X_{t-1}+0.1619X_{t-2}+\epsilon_t-0.4454071\epsilon_{t-1}-0.1012276\epsilon_{t-2} ,\epsilon\sim WN(0,5.222888)$$

%=================================================================%
\end{enumerate}
\end{document}