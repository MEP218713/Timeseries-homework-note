\documentclass[11pt,a4paper]{ctexart}
\usepackage{fontspec}
\defaultfontfeatures{Mapping=tex-text}
\usepackage{xunicode}
\usepackage{xltxtra}
%\setmainfont{???}
\usepackage{amsmath}
\usepackage{amsfonts}
\usepackage{amssymb}
\usepackage{graphicx}
\usepackage{amsthm}
\usepackage{array}
\usepackage{float}   %{H}
\usepackage{booktabs}  %\toprule[1.5pt]
\usepackage[titletoc]{appendix}
%===================%插入代码需要的控制
\usepackage{listings}
\usepackage{xcolor}
\lstset{
	numbers=left, 
	numberstyle= \tiny, 
	keywordstyle= \color{ blue!70},
	commentstyle= \color{red!50!green!50!blue!50}, 
	frame=shadowbox, % 阴影效果
	rulesepcolor= \color{ red!20!green!20!blue!20} ,
	escapeinside=``, % 英文分号中可写入中文
} 
%===================%
\usepackage[left=2cm,right=2cm,top=2cm,bottom=2cm]{geometry}

\newtheorem{theorem}{定理}
\newtheorem{definition}{定义}
\newtheorem*{solution}{解}
\newtheorem*{proofi}{证}
\newtheorem{practice}{题}
\newcommand{\Sum}[3][i]{\sum\limits_{#1=#2}^{#3}}
\newcommand{\Int}[2]{\int_{#1}^{#2}}
\newcommand{\Sample}[3][X]{{#1}_{#2},\dotsi ,{#1}_{#3}}
\newcommand{\Samiid}[4][X]{{#1}_{#2},\dotsi ,{#1}_{#3}~iid\backsim {#4}}
\newcommand{\norm}[1]{\left\Vert #1\right\Vert_{\infty}}
\newcommand{\diff}[3]{\frac{\partial^{#3}{#1}}{\partial {#2}^{#3}}}
\newcommand{\abs}[1]{\left| {#1}\right|}
\newcommand{\normdis}[2]{N(#1,{#2}^2)}

\title{Time Series HomeWork (4)}
\author{钟瑜 \quad 222018314210044}
\date{\today}
\begin{document}
\maketitle
\pagestyle{plain}%设置页码
\begin{enumerate}
%================================================================%	
	
\item[1.]对例2.2中的调和平稳序列求极限
\begin{equation}
\lim_{N\rightarrow\infty}\frac{1}{N}\sum_{t=1}^{N}X_t
\end{equation}
\begin{proofi}
由例2.2知$X_t=b \cos(at+U),t\in\mathbb{Z}$,
其中r.v U在$(-\pi,\pi)$内均匀分布.\\
令$Y_t=U$,显然$\left\lbrace Y_t\right\rbrace $为严平稳遍历时间序列.\\
由定理4.1(2),$X_t=\Phi (Y_{t+1},...,Y_{t+m})=b \cos(at+U)$也是严平稳遍历序列
故由定理4.1(1),
\begin{equation}
\lim_{N\rightarrow\infty}\frac{1}{N}\sum_{t=1}^{N}X_t=\mathbb{E}X_1=\int_{-\pi}^{\pi}b\cos(at+U)=0,a.s.
\end{equation}
\end{proofi}

\item[2.]设$\left\lbrace X_t\right\rbrace $是严平稳序列,对多元函数$\varphi(x_1,...,x_m)$,证明
$$Y_t=\varphi (X_{t+1},..,X_{t+m}),t\in\mathbb{Z}$$
是严平稳序列。
\begin{proofi}
由于$\left\lbrace X_t\right\rbrace $是严平稳序列,故对$\forall n\in\mathbb{N}_+,k\in\mathbb{Z}$,
$$\mathbb{P}(X_1\leq x_1,...,X_n\leq x_n)=\mathbb{P}(X_{1+k}\leq x_{1+k},...,X_{n+k}\leq x_{n+k})$$
故对多元函数$\varphi(x_1,...,x_m)$,
\begin{equation}
\begin{aligned}
\mathbb{P}(Y_1\leq y_1,...,Y_n\leq y_n) 
&= \mathbb{P}(\varphi(X_{1+1},...,X_{1+m})\leq y_1,...,\varphi(X_{n+1},...,X_{n+m})\leq y_n)\\
&= \mathbb{P}((X_{1+1},...,X_{1+m})\leq\varphi^{-1} (y_1),...,(X_{n+1},...,X_{n+m})\leq \varphi(y_n))\\
&= \mathbb{P}((X_{1+1+k},...,X_{1+m+k})\leq\varphi^{-1} (y_{1+k}),...,(X_{n+1+k},...,X_{n+m+k})\leq \varphi(y_{n+k}))\\
&= \mathbb{P}(\varphi(X_{1+1+k},...,X_{1+m+k})\leq y_{1+k},...,\varphi(X_{n+1+k},...,X_{n+m+k})\leq y_{n+k})\\
&= \mathbb{P}(Y_{1+k}\leq y_{1+k},...,Y_{n+k}\leq y_{n+k})\\
\end{aligned}
\end{equation}
即$Y_t=\varphi (X_{t+1},..,X_{t+m}),t\in\mathbb{Z}$是严平稳序列。
\end{proofi}
%=================================================================%
\end{enumerate}

\end{document}