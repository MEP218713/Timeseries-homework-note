\documentclass[11pt,a4paper]{ctexart}
\usepackage{fontspec}
\defaultfontfeatures{Mapping=tex-text}
\usepackage{xunicode}
\usepackage{xltxtra}
%\setmainfont{???}
\usepackage{amsmath}
\usepackage{amsfonts}
\usepackage{amssymb}
\usepackage{graphicx}
\usepackage{amsthm}
\usepackage{array}
\usepackage{float}   %{H}
\usepackage{booktabs}  %\toprule[1.5pt]
\usepackage[titletoc]{appendix}
%===================%插入代码需要的控制
\usepackage{listings}
\usepackage{xcolor}
\setmonofont{Consolas}%字体
\lstset{
	numbers=left, 
	numberstyle= \tiny, 
	keywordstyle= \color{ blue!70},
	commentstyle= \color{red!50!green!50!blue!50}, 
	frame=shadowbox, % 阴影效果
	rulesepcolor= \color{ red!20!green!20!blue!20} ,
	escapeinside=``,% 英文分号中可写入中文
	breaklines=true,
	basicstyle=\ttfamily 
} 
%===================%
\usepackage[left=2cm,right=2cm,top=2cm,bottom=2cm]{geometry}

\newtheorem{theorem}{定理}
\newtheorem{definition}{定义}
\newtheorem*{solution}{解}
\newtheorem{practice}{题}

\title{Time Series HomeWork (9)}
\author{钟瑜 \quad 222018314210044}
\date{\today}
\begin{document}
\maketitle
\pagestyle{plain}%设置页码
\begin{enumerate}
%================================================================%	
	
\item[1.] 设$ \left\lbrace X_t\right\rbrace $为ARMA(1,1)序列,
$ X_t=a_1X_{t-1}+\epsilon_t+b_1\epsilon_{t-1},\epsilon_t\sim WN(0,\sigma^2)$,求$ \gamma_0,\gamma_1,\gamma_2 $.
\begin{solution}
由$ \gamma_k $的递推公式可得
\begin{equation}
\begin{aligned}
\gamma_0 &=a_1\gamma_1+\sigma^2(b_0\psi_0+b_1\psi_1)\\
\gamma_1 &=a_1\gamma_0+\sigma^2b_1\psi_0\\
\gamma_2 &=a_1\gamma_1\\
\end{aligned}
\end{equation}
解得
\begin{equation}
	\begin{aligned}
		\gamma_0 &=\frac{\sigma^2(1+2a_1b_1+b_1^2)}{1-a_1^2}\\
		\gamma_1 &=\frac{\sigma^2(a_1+b_1)(1+a_1b_1)}{1-a_1^2}\\
		\gamma_2 &=\frac{a_1\sigma^2(a_1+b_1)(1+a_1b_1)}{1-a_1^2}\\
	\end{aligned}
\end{equation}

\end{solution}

%=================================================================%
\end{enumerate}
\end{document}