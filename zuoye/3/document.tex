\documentclass[11pt,a4paper]{ctexart}
\usepackage{fontspec}
\defaultfontfeatures{Mapping=tex-text}
\usepackage{xunicode}
\usepackage{xltxtra}
%\setmainfont{???}
\usepackage{amsmath}
\usepackage{amsfonts}
\usepackage{amssymb}
\usepackage{graphicx}
\usepackage{amsthm}
\usepackage{array}
\usepackage{float}   %{H}
\usepackage{booktabs}  %\toprule[1.5pt]
\usepackage[titletoc]{appendix}
%===================%插入代码需要的控制
\usepackage{listings}
\usepackage{xcolor}
\lstset{
	numbers=left, 
	numberstyle= \tiny, 
	keywordstyle= \color{ blue!70},
	commentstyle= \color{red!50!green!50!blue!50}, 
	frame=shadowbox, % 阴影效果
	rulesepcolor= \color{ red!20!green!20!blue!20} ,
	escapeinside=``, % 英文分号中可写入中文
} 
%===================%
\usepackage[left=2cm,right=2cm,top=2cm,bottom=2cm]{geometry}

\newtheorem{theorem}{定理}
\newtheorem{definition}{定义}
\newtheorem*{solution}{解}
\newtheorem{practice}{题}
\newcommand{\Sum}[3][i]{\sum\limits_{#1=#2}^{#3}}
\newcommand{\Int}[2]{\int_{#1}^{#2}}
\newcommand{\Sample}[3][X]{{#1}_{#2},\dotsi ,{#1}_{#3}}
\newcommand{\Samiid}[4][X]{{#1}_{#2},\dotsi ,{#1}_{#3}~iid\backsim {#4}}
\newcommand{\norm}[1]{\left\Vert #1\right\Vert_{\infty}}
\newcommand{\diff}[3]{\frac{\partial^{#3}{#1}}{\partial {#2}^{#3}}}
\newcommand{\abs}[1]{\left| {#1}\right|}
\newcommand{\normdis}[2]{N(#1,{#2}^2)}

\title{Time Series HomeWork (3)}
\author{钟瑜 \quad 222018314210044}
\date{\today}
\begin{document}
\maketitle
\pagestyle{plain}%设置页码
\begin{enumerate}
%================================================================%	
	
\item[1.]如果输入的序列$\left\lbrace X_t\right\rbrace $是由(3.4)定义的线性平稳序列,则从保时线性滤波器H输出的序列$\left\lbrace Y_t\right\rbrace$也是线性平稳序列。

\begin{solution}
设则从保时线性滤波器H输出的序列$\left\lbrace Y_t\right\rbrace$,其中序列$H=\left\lbrace Y_t\right\rbrace $为绝对可和的实数列,要证明其为线性平稳序列,只需证明$\mathbb{E}Y_t$为常数,自协方差函数只与时间差有关。

因为$X_t=\sum^{\infty}_{j=-\infty}a_j\epsilon_{t-j},t\in\mathbb{Z}$为线性平稳序列,故$\mathbb{E}X_t=0$,因此
\begin{equation}
\begin{aligned}
\sum^{\infty}_{j=-\infty}\mathbb{E}|h_jX_{t-j}|=\sum^{\infty}_{j=-\infty}|h_j|\mathbb{E}|X_{t-j}|=0<\infty
\end{aligned}
\end{equation}
由测度论的推论可知,
\begin{equation}
\begin{aligned}
\mathbb{E}Y_t &= \mathbb{E}(\sum_{j=-\infty}^{\infty}h_jX_{t-j}) \\
&= \sum_{j=-\infty}^{\infty}\mathbb{E}(h_jX_{t-j})\\
&= \sum_{j=-\infty}^{\infty}h_j\mathbb{E}(X_{t-j})\\
&= 0
\end{aligned}
\end{equation}
再由
\begin{equation}
\begin{aligned}
\mathbb{E}(X_tX_s)=\sigma^2\sum_{k=-\infty}^{\infty}a_ka_{k+(t-s)}
\end{aligned}
\end{equation}
可得
\begin{equation}
\begin{aligned}
\sum_{j=-\infty}^{\infty}\sum_{i=-\infty}^{\infty}\mathbb{E}|h_jh_iX_{t-j}X_{s-i}|
&=\sum_{j=-\infty}^{\infty}\sum_{i=-\infty}^{\infty}|h_jh_i|\mathbb{E}|X_{t-j}X_{s-i}|\\
&\leq\sum_{j=-\infty}^{\infty}\sum_{i=-\infty}^{\infty}|h_jh_i|
\sqrt{\mathbb{E}|X_{t-j}|^2|X_{s-i}|^2}\\
&= \sum_{j=-\infty}^{\infty}\sum_{i=-\infty}^{\infty}|h_jh_i|
\sigma^2\sum_{-\infty}^{\infty}(a_j)^2\\
&= \sigma^2\sum_{j=-\infty}^{\infty}|h_j|^2\sum_{-\infty}^{\infty}|a_j|^2\\
&< \infty
\end{aligned}
\end{equation}
由测度论的推论可知,
\begin{equation}
\begin{aligned}
\mathbb{E}Y_tY_s &= \mathbb{E}
(\sum_{j=-\infty}^{\infty}h_jX_{t-j})(\sum_{i=-\infty}^{\infty}h_iX_{s-i}) \\
&= \mathbb{E}(\sum_{j=-\infty}^{\infty}\sum_{i=-\infty}^{\infty}h_jh_iX_{t-j}X_{s-i})\\
&= \sum_{j=-\infty}^{\infty}\sum_{i=-\infty}^{\infty}h_jh_i\mathbb{E}(X_{t-j}X_{s-i})\\
&= \sum_{j=-\infty}^{\infty}\sum_{i=-\infty}^{\infty}h_jh_i\sigma^2
\sum_{k=-\infty}^{\infty}a_ka_{k+(t-s)}\\
&= \sigma^2 \sum_{j=-\infty}^{\infty}h_j^2\sum_{k=-\infty}^{\infty}a_ka_{k+(t-s)}
\end{aligned}
\end{equation}
这就说明$\left\lbrace Y_t\right\rbrace$是平稳序列,有自协方差函数
\begin{equation}
\begin{aligned}
\gamma_{l}=\sigma^2 \sum_{j=-\infty}^{\infty}h_j^2\sum_{k=-\infty}^{\infty}a_ka_{k+l},l\in\mathbb{Z}
\end{aligned}
\end{equation}
\end{solution}

\item[2.]证明绝对可和必能推出平方可和。
\begin{solution}
且考虑单边$$\sum_{j=1}^{\infty}|a_j|<\infty$$
那么$$\lim_{j\to\infty}|a_j|=0$$
$$\Rightarrow\lim\frac{a_j^2}{|a_j|}=0$$
根据正项级数比较原则的极限形式可知$$\sum_{j=1}^{\infty}a_j^2<\infty;$$
同理可知由$$\sum_{j=-\infty}^{\infty}|a_j|<\infty$$可推得$$\sum_{j=-\infty}^{\infty}a_j^2<\infty。$$
\end{solution}
%=================================================================%
\end{enumerate}

\end{document}