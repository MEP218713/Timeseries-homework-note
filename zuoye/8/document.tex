\documentclass[11pt,a4paper]{ctexart}
\usepackage{fontspec}
\defaultfontfeatures{Mapping=tex-text}
\usepackage{xunicode}
\usepackage{xltxtra}
%\setmainfont{???}
\usepackage{amsmath}
\usepackage{amsfonts}
\usepackage{amssymb}
\usepackage{graphicx}
\usepackage{amsthm}
\usepackage{array}
\usepackage{float}   %{H}
\usepackage{booktabs}  %\toprule[1.5pt]
\usepackage[titletoc]{appendix}
%===================%插入代码需要的控制
\usepackage{listings}
\usepackage{xcolor}
\setmonofont{Consolas}%字体
\lstset{
	numbers=left, 
	numberstyle= \tiny, 
	keywordstyle= \color{ blue!70},
	commentstyle= \color{red!50!green!50!blue!50}, 
	frame=shadowbox, % 阴影效果
	rulesepcolor= \color{ red!20!green!20!blue!20} ,
	escapeinside=``,% 英文分号中可写入中文
	breaklines=true,
	basicstyle=\ttfamily 
} 
%===================%
\usepackage[left=2cm,right=2cm,top=2cm,bottom=2cm]{geometry}

\newtheorem{theorem}{定理}
\newtheorem{definition}{定义}
\newtheorem*{solution}{解}
\newtheorem{practice}{题}

\title{Time Series HomeWork (8)}
\author{钟瑜 \quad 222018314210044}
\date{\today}
\begin{document}
\maketitle
\pagestyle{plain}%设置页码
\begin{enumerate}
%================================================================%	
	
\item[1.] 对MA(2)模型$ X_t=\epsilon_t-0.66\epsilon_{t-1}+0.765\epsilon_{t-2} $,$ \left\lbrace \epsilon_t \right\rbrace  $是WN(0,4),求$ \gamma_0,\gamma_1,\gamma_2 $和谱密度.
\begin{solution}
由
\begin{equation}
\gamma_k=\sigma^2\sum_{i=0}^{q-k}b_ib_{i+k}
\end{equation}
得
\begin{equation}
\begin{aligned}
\gamma_0 &=4(b_0^2+b_1^2+b_2^2)=8.0833\\
\gamma_1 &=4(b_0b_1+b_1b_2)=-4.6596\\
\gamma_2 &=4b_0b_2=3.06\\
\end{aligned}
\end{equation}
谱密度:
\begin{equation}
	\begin{aligned}
		f(\lambda) =\frac{\sigma^2}{2\pi}|\sum_{j=0}^{2}b_je^{ji\lambda}|^2
		=\frac{4}{2\pi}|1-0.66e^{i\lambda}+0.765e^{2i\lambda}|^2
	\end{aligned}
\end{equation}

\end{solution}


\item[2.]已知平稳序列的自协方差函数
\begin{equation}
	(\gamma_0,\gamma_1,\gamma_2)=(12.4168,-4.7520,5.2000),\gamma_k=0,k\geq 3.
\end{equation}
试为这个平稳序列建立MA(2)模型.
\begin{lstlisting}[language=R]
> library(BB)
> fun <- function(x) #方程组求解函数
+   { 
	+   f <- numeric(length(x)) 
	+   f[1] <-  x[3]*(x[1]*x[1]+x[2]*x[2])-12.4168
	+   f[2] <-  x[3]*(x[1]*x[2]+x[1])+4.7520
	+   f[3] <-  x[3]*x[2]-5.2
	+   f 
	+ } 
> startx <- c(1,2,3)
> result = dfsane(startx,fun,control=list(maxit=2500,trace = FALSE))
> theta = result$par
> theta
[1] -0.6290429  2.2086927  2.3543338
#x[1]=-0.6290429为b_1
#x[2]=2.2086927为b_2
#x[3]=2.3543338为sigma^2
\end{lstlisting}



%=================================================================%
\end{enumerate}
\end{document}